\intro

%
% Введение
%

Биометрические технологии распознавания (идентификации, верификации) личности широко зарекомендовали себя при решении различных задач, связанных с обеспечением повышенного уровня безопасности доступа к информации и различным материальным объектам. В основе технологий лежит свойство уникальности биометрической характеристики человека (индивидуума), используемой в качестве идентификатора. Одной из таких характеристик является изображение радужной оболочки глаза.

Радужная оболочка глаза (РОГ) имеет уникальную, сложную и слабо изменяющуюся со временем структуру, что делает её высокоинформативным и устойчивым биометрическим признаком. Несмотря на то, что свойство уникальности РОГ известно с давних времён, первые новаторские работы (в т.ч. патенты), предлагающие использование радужки в качестве биометрического признака для распознавания, приходятся на период с 1985 по 1998 годы~\cite{flom_safir_patent_1985,daugman_1992,daugman_1993,wildes_1997,wildes_patent_1998,boles_1998}. В качестве входного сигнала было предложено использование изображения РОГ, зарегистрированного цифровой камерой в ближнем инфракрасном (БИК) диапазоне частот спектра электромагнитных волн.

C развитием технических средств регистрации изображения и обработки информации, позволяющая обеспечить наиболее высокую точность распознавания, по сравнению с другими биометрическими методами~\cite{mbgc_2007,mbe_2009,mbc_nist_2010}, технология аутентификации личности по радужной оболочке глаза стала привлекать внимание все большего количества исследовательских групп по всему миру, о чем свидетельствуют данные обзоров технологии, приходящиеся на этот период~\cite{bowyer_survey_2008,ng_overview_2008,labati_overview_2012,bowyer_handbook_2012}. В то же время, одно за другим, стали появляться и первые коммерческие решения в области систем контроля и управления доступом (СКУД), использующие изображение радужки в качестве уникального идентификатора. Среди наиболее известных IriScan, Iridian, Sarnoff, Sensar, LG, Panasonic, OKI, Morpho и другие.

Среди наиболее известных на сегодняшний день биометрических систем, использующих изображение РОГ в качестве уникального идентификатора, можно выделить следующие: системы биометрического паспортного контроля в более чем 10 терминалах аэропортов Великобритании и Амстердама, на границе США и Канады, в 32 наземных, воздушных и морских портах ОАЭ (Совет Сотрудничества Арабских Государств сообщает о 62 триллионах сравнений биометрических шаблонов РОГ за последние 10 лет)~\cite{iris_wiki}; в 2016 году, в рамках программы UIDAI, осуществляемой индийским правительством, изображение радужки было зарегистрировано у более чем 1 млрд жителей страны; изображение РОГ является одной из трёх биометрических модальностей (также лицо и папиллярный узор пальца и ладони), стандартизованных ICAO для применения в электронных паспортах~\cite{icao_2015}.

Одной из основных причин высокого интереса к биометрическим методам аутентификации сегодня является постоянное повышение требований к безопасности, в частности, при проведении финансовых операций, защиты и персонификации пользовательских данных. Большое внимание уделяется в том числе и удобству сервисов, позволяющих отказаться от использования всевозможных паролей, ПИН-кодов, смарт-карт и иных способов защиты. Мобильные устройства, стремительно приобретающие универсальность в аспекте проведения всевозможных транзакций, становятся платформой для развёртывания на них сервисов, использующих методы биометрической аутентификации. Значительная часть смартфонов, появившихся на рынке за последние несколько лет, оборудованы компактными сенсорами для аутентификации пользователя. С каждым годом доля устройств, использующих биометрию для распознавания, увеличивается, а повышение требований к безопасности заставляет производителей прибегать к использованию более сложных средств защиты. Позволяющая обеспечить наивысшую точность и удобство в использовании, технология аутентификации по РОГ привлекает все больше внимания производителей мобильных устройств.

% Актуальность работы
\actualitysection
\actualitytext

% Цели и задачи диссертационной работы
\objectivesection
\objectivetext

% Научная новизна
\noveltysection
\noveltytext

% Теоретическая и практическая значимость
\valuesection
\valuetext

% Результаты и положения, выносимые на защиту
\resultssection
\resultstext

% Степень достоверности и апробация результатов
\approbationsection
\approbationtext

% Публикации
\pubsection
\pubtext

% Личный вклад автора
\contribsection
\contribtext

% Структура и объем диссертации
\structsection
\structtext
